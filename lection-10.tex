\documentclass[aspectratio=169]{beamer}
\usepackage[utf8]{inputenc}
\usepackage[english,russian]{babel}
\usepackage{cancel}
\usepackage{amssymb}
\usepackage{stmaryrd}
\usepackage{cmll}
\usepackage{graphicx}
\usepackage{amsthm}
\usepackage{tikz}
\usepackage{multicol}
\usetikzlibrary{patterns}
\usepackage{chronosys}
\usepackage{proof}
\usepackage{multirow}
\setbeamertemplate{navigation symbols}{}
%\usetheme{Warsaw}

\newtheorem{thm}{Теорема}[section]
\newtheorem{dfn}{Определение}[section]
\newtheorem{lmm}{Лемма}[section]
\newtheorem{exm}{Пример}[section]
\newtheorem{snote}{Пояснение}[section]

\newcommand{\divisible}%
{\mathrel{\lower.2ex%
\vbox{\baselineskip=0.7ex\lineskiplimit=0pt%
\kern6pt \hbox{.}\hbox{.}\hbox{.}}%
}}

\begin{document}

\newcommand\doubleplus{+\kern-1.3ex+\kern0.8ex}
\newcommand\mdoubleplus{\ensuremath{\mathbin{+\mkern-10mu+}}}

\begin{frame}{Невыразимость доказуемости}
\begin{dfn}
$\text{Д}_\mathcal{S} = \{ \ulcorner\alpha\urcorner\ |\ \vdash_\mathcal{S}\alpha \}$; 
$\text{И}_\mathcal{S} = \{ \ulcorner\alpha\urcorner\ |\ \llbracket\alpha\rrbracket_\mathcal{S} = \text{И} \}$
\end{dfn}

\begin{lmm}Пусть $D(\ulcorner\alpha\urcorner) = \ulcorner\alpha(\overline{\ulcorner\alpha\urcorner})\urcorner$ для
любой формулы $\alpha(x)$. Тогда $D$ представима в формальной арифметике.
\end{lmm}


\begin{thm}Если расширение Ф.А. $\mathcal{S}$ непротиворечиво и $D$ представима в нём, то $\text{Д}_\mathcal{S}$ невыразимо в $\mathcal{S}$\end{thm}

\begin{proof}Пусть $\delta(a,p)$ представляет $D$, и пусть $\sigma(x)$ выражает множество $\text{Д}_\mathcal{S}$ (рассматриваемое как
одноместное отношение).

Пусть $\alpha(x) := \forall p.\delta(x,p)\rightarrow\neg\sigma(p)$. Верно ли, что $\ulcorner\alpha\urcorner\in\text{Д}_\mathcal{S}$?
\end{proof}
\end{frame}

\begin{frame}{Неразрешимость формальной арифметики}
\begin{thm}Если формальная арифметика непротиворечива, то формальная арифметика неразрешима\end{thm}
\begin{proof}
Пусть формальная арифметика разрешима. 
Значит, есть рекурсивная функция $f(x)$: $f(x)=1$ тогда и только тогда, 
когда $x \in \text{Д}_\text{Ф.А.}$. То есть, $\text{Д}_\text{Ф.А.}$ выразимо в формальной арифметике.

По теореме о невыразимости доказуемости, 
$\text{Д}_\text{Ф.А.}$ невыразимо в формальной арифметике. Противоречие.
\end{proof}
\end{frame}

\begin{frame}{Теорема Тарского}
\begin{thm}[Тарского о невыразимости истины]
Не существует формулы $\varphi(x)$, что $\llbracket \varphi(\overline{x}) \rrbracket = \text{И}$ (в стандартной интерпретации) тогда и только
тогда, когда $x \in \text{И}_\text{ФА}$. \end{thm}
\begin{proof}
Пусть теория $\mathcal{S}$ --- формальная арифметика + аксиомы: все истинные в стандартной интерпретации формулы.
Очевидно, что $\text{Д}_\mathcal{S} = \text{И}_\mathcal{S} = \text{И}_\text{ФА}$. 
То есть $\text{И}_\text{ФА}$ невыразимо в $\mathcal{S}$.

Пусть $\varphi$ таково, что $\llbracket\varphi(\overline{x})\rrbracket = \text{И}$ при $x \in \text{И}_\text{ФА}$.
Тогда $\vdash\varphi(\overline{x})$, если $x \in \text{И}_\text{ФА}$ и $\vdash\neg\varphi(\overline{x})$, если $x \notin\text{И}_\text{ФА}$.

Тогда $\text{И}_\text{ФА}$ выразимо в $\mathcal{S}$. Противоречие.
\end{proof}

\pause
Однако, если взять $D = \mathbb{R}$, истина становится выразима (алгоритм Тарского).
\end{frame}

\begin{frame}{}
\LARGE\begin{center}Теория множеств\end{center}
\end{frame}

\begin{frame}{Теория множеств}
\begin{enumerate}
\item Георг Кантор: 1877 год, <<наивная теория множеств>>. Множество --- это «объединение в одно
целое объектов, хорошо различаемых нашей интуицией или нашей мыслью».\pause
\item Неограниченный принцип абстракции $\{ x\ |\ P(x)\}$ \pause
\item Парадокс Бурали-Форти (1895, Кантор). Парадокс Рассела: $X := \{ x\ |\ x \notin x\}$; $X\in X$?\pause
\item Вариант решения парадокса: а, может, запретить все <<опасные>> ситуации? \pause
\item Аксиоматика Цермело --- 1908 год, оставим только то, что используют математики. \pause
\item Что такое множество? Неформально мы понимаем, формально:\pause

\begin{dfn} Теория множеств --- теория первого порядка,
с дополнительным нелогическим двухместным предикатным символом $\in$, и следующими 
дополнительными нелогическими аксиомами и схемами аксиом.
\end{dfn}
\end{enumerate}
\end{frame}

\begin{frame}{Аксиоматика ZF, равенство}
\begin{dfn} Равенство <<по Лейбницу>>: объекты равны, если неразличимы.\end{dfn} Если нечто ходит как утка, выглядит как 
утка и крякает как утка, то это утка.\pause
\begin{dfn} Принцип объёмности: объекты равны, если состоят из одинаковых частей\end{dfn}\pause

\begin{dfn} $A \subseteq B \equiv \forall x.x \in A \rightarrow x \in B$ \\\pause
 $A = B \equiv A \subseteq B \with B \subseteq A$ \end{dfn}\pause
\begin{dfn} Аксиома равенства: равные множества содержатся в одних и тех же множествах. 
$\forall x. \forall y. \forall z. x = y \with x \in z \rightarrow y \in z$.
\end{dfn}
\end{frame}

\begin{frame}{Аксиоматика ZF, конструктивные аксиомы}
\begin{dfn} Аксиома пустого. Существует пустое множество $\varnothing$. $$\exists s.\forall t.\neg t \in s$$ \end{dfn}\pause
\begin{dfn} Аксиома пары. Существует $\{a,b\}$.
Каковы бы ни были два множества $a$ и $b$, существует множество, состоящее 
в точности из них. 

$$\forall a.\forall b.\exists s.a \in s \with b \in s \with \forall c.c \in s \rightarrow c = a \vee c = b$$ \end{dfn}
\end{frame}

\begin{frame}{Аксиоматика ZF, конструктивные аксиомы 2}
\begin{dfn} Аксиома объединения: существует $\cup x$. 
Для любого непустого множества $x$ найдётся такое множество, состоящее в точности
из тех элементов, из которых состоят элементы $x$. 
$$\forall x.(\exists y.y \in x) \rightarrow \exists p.\forall y.y \in p \leftrightarrow \exists s.y \in s \with s \in x$$
 \end{dfn}\pause
\begin{dfn} Аксиома степени: существует $\mathcal{P}(x)$.
Каково бы ни было множество $x$, существует множество, содержащее в точности
все возможные подмножества множества $x$.
$$\forall x.\exists p.\forall y.y \in p \leftrightarrow y \subseteq x$$
\end{dfn}
\end{frame}

\begin{frame}{Аксиоматика ZF. Схема аксиом выделения}
\begin{dfn} Схема аксиом выделения: существует $\{ t \in x\ |\ \varphi(t)\}$.
Для любого множества $x$ и любой формулы от одного аргумента $\varphi(y)$
($b$ не входит свободно в $\varphi$), найдется $b$, в которое
входят те и только те элементы из множества $x$, что $\varphi(y)$ истинно.

$$\forall x.\exists b.\forall y.y \in b \leftrightarrow (y \in x \with \varphi(y))$$
\end{dfn}
\end{frame}

\begin{frame}{Немного теорем}
\begin{thm}Для любого множества $X$ существует множество $\{X\}$, содержащее в точности $X$.\end{thm}\pause
\begin{proof}Воспользуемся аксиомой пары: $\{X,X\}$\end{proof}\pause
\begin{thm}Пустое множество единственно.\end{thm}\pause
\begin{proof}Пусть $\forall p.\neg p \in s$ и $\forall p.\neg p \in t$.
Тогда $s \subseteq t$ и $t \subseteq s$.\end{proof}\pause
\begin{thm}Для двух множеств $s$ и $t$ существует множество, являющееся их пересечением.\end{thm}\pause
\begin{proof}$s \cap t = \{ x\in s\ |\ x \in t\}$\end{proof}
\end{frame}

\begin{frame}{Упорядоченная пара}
\begin{dfn}{Упорядоченная пара.}
Упорядоченной парой двух множеств $a$ и $b$ назовём
$\{\{a\},\{a,b\}\}$, или $\langle{}a,b\rangle$
\end{dfn}

\begin{thm}
Упорядоченную пару можно построить для любых множеств.
\end{thm}
\begin{proof}Применить аксиому пары, теорему о существовании $\{X\}$, аксиому пары.\end{proof}

\begin{thm}
$\langle{}a,b\rangle = \langle{}c,d\rangle$ тогда и только тогда,
когда $a = c$ и $b = d$.
\end{thm}
\end{frame}

\begin{frame}{Аксиома бесконечности}
\begin{dfn}Инкремент: $x' \equiv x \cup \{x\}$\end{dfn}\pause
\begin{dfn}Аксиома бесконечности. Существует $N: \varnothing \in N \with \forall x.x \in N\rightarrow x' \in N$\end{dfn}\pause

В $N$ есть всевозможные множества вида $\varnothing$\pause, $\{\varnothing\}$\pause, $\{\varnothing,\{\varnothing\}\}$, \pause
$\{\varnothing,\{\varnothing\},\{\varnothing,\{\varnothing\}\}\}$, \dots\pause
\\\vspace{0.5cm}
(неформально) $\omega = \{\varnothing, \varnothing', \varnothing'', \dots\}$. \pause
Тогда $N_1 = \omega\cup\{\omega,\omega',\omega'',\dots\}$ подходит.
\end{frame}

\begin{frame}{Полный порядок (вполне упорядоченные множества)}
\begin{dfn}[отношения нестрогого порядка]
\begin{enumerate}
\item Частичный: рефлексивность ($a \preceq a$), антисимметричность ($a \preceq b \rightarrow b \preceq a\rightarrow a=b$),
транзитивность ($a \preceq b \rightarrow b \preceq c \rightarrow a \preceq c$).\pause
\item Линейный: частичный + $\forall a.\forall b.a \preceq b \vee b \preceq a$.\pause
\item Полный: линейный + в любом непустом подмножестве есть наименьший элемент.\pause
\end{enumerate}
\end{dfn}

\begin{exm}$\mathbb{Z}$ не вполне упорядочено: в $\mathbb{Z}$ нет наименьшего.\end{exm}\pause
\begin{exm}Отрезок $[0,1]$ не вполне упорядочен: $(0,1)$ не имеет наименьшего.\end{exm}\pause
\begin{exm}$\mathbb{N}$ вполне упорядочено.\end{exm}
\end{frame}

\begin{frame}{Отношения строгого и нестрогого порядка}
Для отношения строгого (нестрогого) порядка легко найти парное отношение нестрогого (строгого) порядка.
\begin{center}\begin{tabular}{l|l}
Строгий порядок & Нестрогий порядок \\\hline
$a \prec b$ & $a \prec b \vee a = b$\\
$a \preceq b \with a \ne b$ & $a \preceq b$\\\hline
$A \in B$ & $A \in B \vee A = B$
\end{tabular}\end{center}

\begin{dfn}[полный строгий порядок]
$A$ вполне упорядочено отношением $(\prec)$, если:
\begin{enumerate}
\item при всех $a,b \in A$ выполнено либо $a \prec b$, либо $b \prec a$, либо $a = b$;
\item в любом $S \subseteq A$ и $S \ne \varnothing$ найдётся $n \in S$, что $\forall x.x \in S \rightarrow n = x \vee n \prec x$.
\end{enumerate}
\end{dfn}
\end{frame}

\begin{frame}{Ординалы (порядковые числа)}
\begin{dfn}Транзитивное множество $X$: $\forall x.\forall y.x \in y \with y \in X \rightarrow x \in X$.\end{dfn}\pause
\begin{dfn}Ординал (порядковое число) --- вполне упорядоченное отношением $(\in)$ транзитивное множество.\end{dfn}\pause
\begin{exm}Ординалы: $\varnothing$, \pause $\varnothing'$, \pause $\varnothing''$, \dots\end{exm}\pause
\begin{dfn}Предельный ординал: такой $x$, что $x \ne \varnothing$ и нет $y: y' = x$\end{dfn}\pause
\begin{dfn}Ординал $x$ конечный, если он сам не предельный и нет предельного, меньшего его.\end{dfn}\pause
\begin{thm}Если $x,y$ --- ординалы, то $x = y$, или $x\in y$, или $y \in x$.\end{thm}
\end{frame}
\begin{frame}{Предельные ординалы, $\omega$}
\begin{dfn}$\omega$ --- наименьший предельный ординал.\end{dfn}\pause
\begin{thm}$\omega$ существует.\end{thm}\pause
\begin{proof}Пусть $\omega = \{ x \in N\ |\ x\text{ конечен}\}$. Тогда:
\begin{itemize}
\item меньше $\omega$ предельных нет: если $\theta$ таков, что $\theta \in \omega$, тогда $\theta$ конечен.\pause
\item $\omega$ предельный: Пусть $\theta$ таков, что $\theta' = \omega$. Тогда $\theta$ конечен и $\theta'$ тоже конечен.
\end{itemize}\end{proof}
\begin{exm}$\omega'$ --- тоже ординал.\end{exm}
\end{frame}

\begin{frame}{Порядковый тип}
\begin{dfn}[неформальное определение]Порядковый тип множества --- некоторое свойство, общее для всех множеств, 
изоморфных относительно биективных отображений, сохраняющих порядок.\end{dfn}

\begin{dfn}Порядковый тип вполне упорядоченного множества $\langle S, (\preceq)\rangle$ --- ординал $A$, для которого есть биективное отображение $f: S \rightarrow A$, сохраняющее порядок:
$a \preceq b$ тогда и только тогда, когда $f(a) \le f(b)$\end{dfn}

\begin{exm}Множество $\mathbb{Z}$ не имеет порядкового типа (в смысле определения через ординалы): оно не вполне упорядочено.\end{exm}
\end{frame}

\begin{frame}{Операции над ординалами}
\begin{dfn}$a + b$ --- порядковый тип $a \uplus b$ (отмеченного объединения), причём $x_a < y_b$ при любых
$x \in a$ и $y \in b$\end{dfn}

\begin{dfn}$a \cdot b$ --- порядковый тип $a \times b$, произведение упорядочено лексикографически: $\langle x_1, y_1 \rangle < \langle x_2, y_2 \rangle$, 
если $y_1 < y_2$ или $y_1 = y_2$ и $x_1 < x_2$.\end{dfn}

\begin{exm}$\overline{3} + \overline{4}$: порядковый тип множества $\{0_a, 1_a, 2_a, 0_b, 1_b, 2_b, 3_b\}$, то есть $\overline{7}$

$\omega \cdot \omega$: порядковый тип всех натуральных точек плоскости, $\{\langle 0,0 \rangle, \dots, \langle 0,100\rangle, \dots, \langle 100,0\rangle, \dots\}$\end{exm}
\end{frame}

\begin{frame}{Операции над ординалами --- как вычислять}
\begin{dfn}$\text{upb } x$ --- верхняя грань множества ординалов, $\text{upb }x = \bigcup_{a \in x} a$.\end{dfn} \pause
\begin{exm}$\text{upb } \{ \varnothing', \varnothing'', \varnothing'''' \} = \varnothing' \cup \varnothing'' \cup \varnothing'''' =
\{ \varnothing \} \cup \{ \varnothing, \{ \varnothing \} \} \cup \{ \varnothing, \{ \varnothing \}, \{ \varnothing, \{ \varnothing \} \}, \{ \varnothing, \{ \varnothing \}, \{ \varnothing, \{ \varnothing \} \} \} \}
= \pause \varnothing''''$\end{exm} \pause

\begin{thm}
$$a + b \equiv \left\{ \begin{array}{rl} 
   a, & b \equiv \varnothing\\
   (a + c)', & b \equiv c'\\
   \text{upb } \{ a+c \mid c \prec b \}, &\mbox{$b$ --- предельный ординал }\end{array}\right.$$\end{thm}

\begin{exm}$\omega + 1 = \omega \cup \{\omega\}$; \pause $1 + \omega = \text{upb }\{ 1+\varnothing, 1+1, 1+2, \dots \} \pause = \omega$\end{exm}
\end{frame}

\begin{frame}{Ещё операции над ординалами}
\begin{thm}
$$a \cdot b \equiv \left\{ \begin{array}{rl} 
   0, & b \equiv \varnothing\\
   (a \cdot c) + a, & b \equiv c'\\
   \text{upb } \{ a \cdot c \mid c \prec b \}, &\mbox{$b$ --- предельный ординал }\end{array}\right.$$
\end{thm}
\pause
\begin{dfn}
$$a ^ b \equiv \left\{ \begin{array}{rl} 
   1, & b \equiv \varnothing\\
   (a ^ c) \cdot a, & b \equiv c'\\
   \text{upb } \{ a^c \mid c \prec b \}, &\mbox{$b$ --- предельный ординал }\end{array}\right.$$
\end{dfn}
\pause
\begin{exm}$\omega \cdot \omega = \text{upb }\{\omega \cdot 0, \omega \cdot 1,\omega\cdot 2, \omega\cdot 3, \dots\} = \text{upb }\{0, \omega,\omega\cdot 2, \omega\cdot 3, \dots\}$\end{exm}
\end{frame}

\begin{frame}{Ординалы (порядковые числа) и порядок}
\begin{exm}\begin{itemize}
\item Добавить элемент перед бесконечностью: $\mathbb{N}$ и $\mathbb{N}_0$.
\pause
$1 + \omega = \omega$. \pause
\item Добавить элемент после бесконечности $(+\infty)$. \pause $\omega + 1 \ne \omega$ %\pause
%\item 
\end{itemize}\end{exm} 
\end{frame}

\end{document}
