\documentclass[11pt,a4paper,oneside]{scrartcl}
\usepackage[utf8]{inputenc}
\usepackage[english,russian]{babel}
\usepackage[top=1cm,bottom=1cm,left=1cm,right=1cm]{geometry}

\begin{document}
\pagestyle{empty}

\begin{center}
{\large\scshape\bfseries Программа курса <<Математическая логика>>}\\
{\large\scshape Вопросы к зачёту и экзамену.}\\
\itshape ИТМО, группы M3232--M3239, осень 2025 г.
\end{center}

%\vspace{0.3cm}

\begin{enumerate}
\item Исчисление высказываний. Предметный язык и язык исследователя (метаязык). 
Язык исчисления высказываний. Оценка высказываний, общезначимость, следование.
Доказуемость, гипотезы (контекст), выводимость. Корректность, полнота, противоречивость и 
непротиворечивость (эквивалентные формулировки). Теорема о дедукции для исчисления высказываний.
\item Теорема о полноте исчисления высказываний. Условное отрицание. 14 лемм о связках. 
Лемма об устранении посылок. Доказательство теоремы.
\item Топологические пространства. Определение. Примеры (топология стрелки, Зарисского, 
топология на деревьях). Открытые и замкнутые множества. Связность. Компактность. Непрерывные функции.
\item Гильбертов вывод и натуральный вывод. Интуиционистское исчисление высказываний.
Доказательства чистого существования. BHK-интерпретация. 
Закон исключённого третьего, принцип взрыва, связь с КИВ и ИИВ.
Решётки. Дистрибутивная решётка. Псевдодополнение. Булевы и псевдобулевы алгебры.
\item Алгебра Линденбаума. Полнота интуиционистского исчисления высказываний в псевдобулевых 
алгебрах. Модели Крипке. Вынужденность. Сведение моделей Крипке к псевдобулевым алгебрам. 
Нетабличность ИИВ (формулировка теоремы).
\item Гёделева алгебра. Операция $\Gamma(A)$. Дизъюнктивность ИИВ.
Подрешётка. Разрешимость интуиционистского исчисления высказываний.
\item Категорические силлогизмы. Термины, предикат, субъект, фигуры, модусы (сильные, слабые, неправильные),
ограничения, контрпримеры на ограничения.
Исчисление предикатов. Язык исчисления предикатов.
Метаязык, сокращения записи.
Вхождения, свободные вхождения, подстановка, свобода для подстановки.
Теория доказательств для исчисления предикатов, выводимость.
Доказательства свойств категорических силлогизмов (формулировка свойств сильных и слабых
силлогизмов на языке исчисления предикатов, их доказательство).
\item Теория моделей исчисления предикатов (предметное множество, оценка).
Функции (предикаты) и функциональные (предикатные) символы.
Общезначимость, следование.
Теорема о дедукции в исчислении предикатов. Отличия от исчисления высказываний.
Лемма о перестановке подстановки и оценки. Теорема о корректности исчисления предикатов.
\item Непротиворечивые множества формул (с кванторами и бескванторные).
Пополнение множества формул.
Существование моделей у непротиворечивых множеств формул в бескванторном исчислении предикатов.
\item Поверхностные кванторы (предварённая форма). Эквивалентность формул формулам с поверхностными кванторами
(формулировка теоремы).
Сколемизация.
Теорема Гёделя о полноте исчисления предикатов. 
Полнота исчисления предикатов.
\item Машина Тьюринга. Разрешимость теории, примеры. Задача об останове, её неразрешимость. 
Неразрешимость исчисления предикатов.
\item Представление чисел через натуральные (целые, рациональные, вещественные).
Аксиоматика Пеано. Арифметические операции (сложение, умножение, возведение в степень) в аксиоматике Пеано.
Доказательство коммутативности сложения.
Порядок теории (0, 1, 2). Теории первого порядка. Формальная арифметика. Доказательство $a=a$.
Арифметизация математики, формализация категорических силлогизмов, предложенная Лейбницем.
\item Примитивно-рекурсивные и рекурсивные функции. 
Функции вычисления простых чисел. Частичный логарифм.
Выразимость отношений и представимость функций в формальной арифметике. Характеристические функции.
Функция Аккермана. Доказательство невозможности выражения функции Аккермана в примитивно-рекурсивных функциях.
\item Представимость примитивов $N$, $Z$, $U$, $S$. Бета-функция Гёделя. 
Представимость $R$ и $M$, представимость рекурсивных функций в формальной арифметике.
\item Гёделева нумерация. Рекурсивность представимых в формальной арифметике функций.
Функции $W_1$ и $W_2$.
\item Непротиворечивость (эквивалентные определения), $\omega$-не\-про\-ти\-во\-ре\-чи\-вость. 
Первая теорема Гёделя о неполноте арифметики.
Формулировка первой теоремы Гёделя о неполноте арифметики в форме Россера. 
Синтаксическая и семантическая неполнота арифметики.
Неполнота расширений формальной арифметики.
Ослабленные варианты: арифметика Пресбургера, система Робинсона.
\item Вторая теорема Гёделя о неполноте арифметики, $Consis$. 
Лемма об автоссылках. Условия Гильберта-Бернайса-Лёба. Неразрешимость формальной арифметики. 
Теорема Тарского о невыразимости истины.
\item Теория множеств. Определения равенства. Парадокс брадобрея. Аксиоматика Цермело-Френкеля. 
Конструктивные аксиомы (пустого, пары, объединения, множества подмножеств, выделения).
Частичный, линейный, полный порядок. Ординальные числа, аксиома бесконечности.
Конечные ординалы, предельные ординалы, доказательство существования ординала $\omega$, 
операции над ординалами (варианты определения), факты об операциях над ординалами 
(выполнены ли ассоциативность и коммутативность операций). Связь ординалов и упорядочений.
\item Аксиомы фундирования и подстановки. Кардинальные числа, мощность множеств, операции над 
кардинальными числами (сложение, умножение, возведение в степень). Теорема Кантора-Бернштейна, 
теорема Кантора. 
\item Мощность модели. Элементарные подмодели. Теорема Лёвенгейма-Сколема, парадокс Сколема.
\item Аксиома выбора, альтернативные формулировки (лемма Цорна, теорема Цермело, существование
частичной обратной), доказательство переходов (кроме доказательства леммы Цорна).
\item Применение аксиомы выбора: эквивалентность определений пределов (по Коши и по Гейне).
Теорема Диаконеску. Ослабленные варианты (счётный выбор и зависимый выбор), универсум фон Неймана.
Аксиома конструктивности.
\item Индукция и полная индукция. Наследственные множества. Трансфинитная индукция
(аналоги полного и обычного варианта математической индукции). Доказательство $a \cdot a = a$ при $a \ge \aleph_0$.
\item Система $S_\infty$, степень и порядок доказательства. 
Правило сечения, теорема об устранении сечений. 
Доказательство непротиворечивости формальной арифметики.
\item Сколемизация. Эрбранов универсум, основные термы, эрбранова интерпретация,
система дизъюнктов, основные примеры, система основных примеров, теорема Гёделя о компактности,
теорема Эрбрана. Правило резолюции (для исчисления высказываний и для исчисления предикатов),
задачи унификации, уравнения в алгебраических термах, наибольший общий унификатор.
Общая формулировка метода резолюции. SMT-решатели.
\item Лямбда-исчисление. Пред- и лямбда-термы. Альфа-эквивалентность, бета-редукция, бета-эквивалентность.
Теорема Чёрча-Россера (формулировка). Нормальная форма и её единственность (с доказательством).
Представление истины и лжи, чёрчевские нумералы, арифметические функции (сложение, умножение, вычитание).
Комбинатор неподвижной точки. Импликационный фрагмент ИИВ. Замкнутость импликационного фрагмента ИИВ 
(формулировка). Типизация лямбда-исчисления по Чёрчу и по Карри. Гильбертов вывод и комбинаторы.
\item Модальная логика, системы K, K4, T, S4, S5. Линейная темпоральная логика. Построение формулы для выбранного
инварианта (например, условия на семафоры критической секции для двух потоков).
Проверка на моделях, постановка задачи. Система переходов. Автоматы Бюхи. 
Построение автомата Бюхи по данной формуле ЛТЛ. Схема алгоритма, проверяющего с помощью моделей 
соответствие алгоритма утверждению в ЛТЛ.
\end{enumerate}

\end{document}
