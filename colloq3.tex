\documentclass[11pt,a4paper,oneside]{scrartcl}
\usepackage[utf8]{inputenc}
\usepackage[english,russian]{babel}
\usepackage[top=1cm,bottom=1cm,left=1cm,right=1cm]{geometry}

\begin{document}
\pagestyle{empty}

\begin{center}
{\large\scshape\bfseries Программа курса <<Математическая логика>>}\\
{\large\scshape Вопросы к третьему коллоквиуму.}\\
\itshape ИТМО, группы M3232--M3239, осень 2025 г.
\end{center}

%\vspace{0.3cm}

\begin{enumerate}
\item Теория множеств. Определения равенства. Парадокс брадобрея. Аксиоматика Цермело-Френкеля. 
Конструктивные аксиомы
(пустого, пары, объединения, множества подмножеств, выделения).
Частичный, линейный, полный порядок. Ординальные числа, аксиома бесконечности. 
\item Конечные ординалы, предельные ординалы, существование ординала $\omega$, операции над ординалами, 
факты об операциях над ординалами (сравнение $a + b$ и $b + a$, $a\cdot b$ и $b \cdot a$). 
Связь ординалов и упорядочений. Аксиомы фундирования и подстановки.
\item Кардинальные числа, мощность множеств, операции над кардинальными числами
(сложение, умножение, возведение в степень). Теорема Кантора-Бернштейна, теорема Кантора. 
\item Мощность модели. Элементарные подмодели. Теорема Лёвенгейма-Сколема, парадокс Сколема.
\item Аксиома выбора, альтернативные формулировки (лемма Цорна, теорема Цермело, существование
частичной обратной), доказательство переходов (кроме доказательства леммы Цорна).
\item Применение аксиомы выбора: эквивалентность определений пределов (по Коши и по Гейне).
Теорема Диаконеску. Ослабленные варианты (счётный выбор и зависимый выбор), универсум фон Неймана.
Аксиома конструктивности.
\item Индукция и полная индукция. Наследственные множества. Трансфинитная индукция
(аналоги полного и обычного варианта математической индукции). 
Применение трансфинитной индукции. Система $S_\infty$, степень и порядок доказательства. 
Правило сечения, теорема об устранении сечений. Доказательство непротиворечивости формальной арифметики.
\item  Сколемизация. Эрбранов универсум, основные термы, эрбранова интерпретация,
система дизъюнктов, основные примеры, система основных примеров, теорема Гёделя о компактности,
теорема Эрбрана. Правило резолюции (для исчисления высказываний и для исчисления предикатов),
задачи унификации, уравнения в алгебраических термах, наибольший общий унификатор.
Общая формулировка метода резолюции.
SMT-решатели.
\end{enumerate}
\end{document}
