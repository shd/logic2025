\documentclass[aspectratio=169]{beamer}
\usepackage[utf8]{inputenc}
\usepackage[english,russian]{babel}
\usepackage{cancel}
\usepackage{amssymb}
\usepackage{stmaryrd}
\usepackage{cmll}
\usepackage{graphicx}
\usepackage{amsthm}
\usepackage{tikz}
\usepackage{multicol}
\usetikzlibrary{patterns}
\usepackage{chronosys}
\usepackage{proof}
\usepackage{multirow}
\setbeamertemplate{navigation symbols}{}
%\usetheme{Warsaw}

\newtheorem{thm}{Теорема}[section]
\newtheorem{dfn}{Определение}[section]
\newtheorem{lmm}{Лемма}[section]
\newtheorem{exm}{Пример}[section]
\newtheorem{snote}{Пояснение}[section]

\newcommand{\divisible}%
{\mathrel{\lower.2ex%
\vbox{\baselineskip=0.7ex\lineskiplimit=0pt%
\kern6pt \hbox{.}\hbox{.}\hbox{.}}%
}}

\begin{document}

\newcommand\doubleplus{+\kern-1.3ex+\kern0.8ex}
\newcommand\mdoubleplus{\ensuremath{\mathbin{+\mkern-10mu+}}}


\begin{frame}{}
\begin{center}\LARGE Метод резолюции\end{center}
\end{frame}

\begin{frame}{Как найти доказательство для формулы исчисления предикатов}

%\begin{center}\LARGE Метод резолюции\end{center}
%\vspace{0.5cm}
%\footnotesize\flushright
%--- Это же п-проблема Бен Б-бецалеля. К-калиостро же доказал, что она н-не имеет р-решения. 
%--- Мы сами знаем, что она не имеет решения, --- сказал Хунта, немедленно ощетиниваясь. --- Мы хотим знать, как ее решать. 
%--- К-как-то ты странно рассуждаешь, К-кристо… К-как же искать решение, к-когда его нет? Б-бессмыслица какая-то… 
%--- Извини, Теодор, но это ты очень странно рассуждаешь. 
%Бессмыслица --- искать решение, если оно и так есть. 
%Речь идет о том, как поступать с задачей, которая решения не имеет. 
%Это глубоко принципиальный вопрос, который, 
%как я вижу, тебе, прикладнику, к сожалению, не доступен.
%\flushright{\itshape Братья Стругацкие, <<Понедельник начинается в субботу>>}
%\vspace{0.25cm}

\normalsize\flushleft

\begin{itemize}
\item Задачи проверки истинности и доказуемости формул исчисления предикатов неразрешимы.
\item Однако, эти задачи неплохо решаются людьми в практических ситуациях.
\item Налицо типичная постановка задачи искусственного интеллекта --- можно попробовать что-то придумать.
\end{itemize}

\end{frame}

\begin{frame}{Что можно сделать для разрешимости исчисления предикатов?}
\begin{itemize}
\item По теореме о полноте можем рассматривать $(\models)$ вместо $(\vdash)$. 
Напомним: $\models \alpha$, если для всех $M = \langle D, F, P, E \rangle$ выполнено $M \models \alpha$.\pause
\vspace{0.3cm}

\item Что мешает проверке истинности:
\begin{enumerate}
\item слишком сложные формулы --- кванторы по бесконечным множествам;
\item слишком большое разнообразие $D$, включая несчётные;
\item даже $D = \mathbb{N}$ в формальной арифметике представляет проблему.
\end{enumerate}\pause
\vspace{0.3cm}

\item Будем последовательно бороться:
\begin{enumerate}
\item упростим формулу (борьба с кванторами);
\item заменим произвольное $D$ на какое-то рекурсивно-перечислимое множество, устроенное некоторым фиксированным образом (борьба с разнообразием $D$);
\item устроим правильный перебор, позволяющий быстро находить решения, если они есть (борьба с бесконечностью $D$).
\end{enumerate}
\end{itemize}
\end{frame}

\begin{frame}{Шаги рассуждения}
\begin{enumerate}
\item \color{black}Упростим формулу --- избавимся от кванторов.
\item \color{gray}Заменяем модель ($D$ и значения функциональных и предикатных символов).
\item \color{gray}Правильный перебор
\end{enumerate}
\end{frame}


\begin{frame}{Упрощаем формулу $\alpha$, сколемизация}
\begin{enumerate}
\item Для любой $\alpha$ найдётся $\beta$ с поверхностными кванторами, что $\vdash\alpha \leftrightarrow \beta$.
\emph{В качестве примера} пусть в $\beta$ оказались чередующиеся кванторы:
$$\beta := \forall x_1.\exists x_2.\forall x_3.\exists x_4\dots \forall x_{n-1}.\exists x_n.\varphi$$

\item Исходная задача: проверка $\vdash\alpha$. Это эквивалентно $\vdash\beta$. Эквивалентно $\models\beta$.
То есть, при любом $D$:
\begin{itemize}
\item при любом $x_1$ найдётся такой $x_2$, что ...
\item при любом $x_3$ найдётся $x_4$, что ... (и т.д.) ...
\item что найдётся $x_n$, что $\varphi$ истинен.
\end{itemize}

\item Заменим $x_{2k}$ \emph{функциями Сколема} $e_{2k}(x_1,x_3,\dots,x_{2k-1})$.
Получим: $$\eta := \forall x_1.\forall x_3\dots\forall x_{n-1}.\varphi[x_2:=e_2(x_1), x_4:=e_4(x_1,x_3), \dots, x_n := e_n(x_1,x_3,\dots,x_{n-1})]$$

%То есть, при любом $D$ найдутся такие $e_i$, что при любых $x_1, x_3, \dots x_{n-1}$ 
%выполнено $\llbracket\varphi\rrbracket^{x_2:=e_2(x_1), x_4:=e_4(x_1,x_3), \dots, x_n := e_n(x_1,x_3,\dots,x_{n-1})} = \text{И}$.

\item Сколемизация сохраняет выполнимость: 
$(\Rightarrow)$ если $\beta$ истинна, то рассмотрим все $x_1,x_2,\dots$, что $\varphi$ истинна ---
и положим $e_k(x_1,x_3,\dots,x_{k-1}) := x_k$;
$(\Leftarrow)$ если $\eta$ истинна, то $\beta$ истинна в той же оценке.

\end{enumerate}
\end{frame}

\begin{frame}{Сколемизация, избавляемся от чередований кванторов}
\begin{enumerate}
\item Было: $\beta := \forall x_1.\exists x_2.\forall x_3.\exists x_4\dots \forall x_{n-1}.\exists x_n.\varphi$
($n$ чередований кванторов).
\item Сколемизация не сохраняет общезначимость: $\vdash\forall x.\exists y.y > x$, но
$\forall x.e(y) > x$ ложно при $e(y) := x$.
\item Поэтому мы проверяем иное свойство: при любом $D$ найдутся $e_i$, что 
$$\eta := \forall x_1.\forall x_3\dots\forall x_{n-1}.\varphi[x_2:=e_2(x_1), \dots, x_n := e_n(x_1,x_3,\dots,x_{n-1})]$$

%при любых $x_1, x_3, \dots x_{n-1}$ 
%$\llbracket\varphi\rrbracket^{x_2:=e_2(x_1), x_4:=e_4(x_1,x_3), \dots, x_n := e_n(x_1,x_3,\dots,x_{n-1})} = \text{И}$\\
Два чередования: при любом $D$ --- найдутся $e_{2k}$ --- что $\forall x_1.\forall x_3.\dots\forall.x_{n-1}.\varphi$
\item Как бы убрать и это чередование? С помощью отрицания --- рассмотрим $\alpha' := \neg\alpha$ и
сколемизированное представление для неё:
$$\eta' := \forall x_1.\forall x_3\dots\forall x_{m-1}.\varphi'[x_2:=e'_2(x_1), \dots, x_m := e'_n(x_1,x_3,\dots,x_{m-1})]$$

Тогда $\models\alpha$ соответствует невыполнимости $\alpha'$, то есть невыполнимости $\eta'$.
%\item При любом $D$, при любых $e'_{2k}$, при любых $x_1,x_3,\dots$ не выполнено $\varphi'$.
\item То есть, $\models \alpha$ тогда и только тогда, когда
при любом $D$ --- при любых $e'$ --- найдутся $x_1, x_3, \dots x_{m-1}$, что ложно

$$\varphi'[x_2:=e'_2(x_1), x_4:=e'_4(x_1,x_3), \dots, x_m := e'_n(x_1,x_3,\dots,x_{m-1})]$$
\end{enumerate}
\end{frame}

\begin{frame}{Преобразуем формулу в КНФ}

\begin{dfn}КНФ ($c$ конъюнктов, в каждом $d(c)$ дизъюнктов, каждый --- предикатный символ с возможным отрицанием):
$$\zeta := \forall x_1.\forall x_3\dots\forall x_{n-1}.
    \bigwedge_{k=\overline{1,c}}\left(\bigvee_{i = \overline{1,d(c)}}\delta^k_i\right)$$
при этом
$$\delta^k_i := P^k_i(\theta^k_{i,1},\dots,\theta^k_{i,a(k,i)})\text{ или }\delta^k_i := \neg P^k_i(\theta^k_{i,1},\dots,\theta^k_{i,a(k,i)})$$
\end{dfn}

\begin{thm} Для любой $\varphi$ найдётся эквивалентная ей формула в КНФ.\end{thm}
%$$\delta := \forall x_1.\forall x_3\dots\forall x_{n-1}.\bigwedge_{k=\overline{1,c}}\left(\bigvee_{i = \overline{1,d(c)}} \{\neg\}^k_i P^k_i(\overline{\theta^k_i})\right)$$
%\item КНФ ($c$ конъюнктов, в каждом $d(c)$ дизъюнктов, каждый --- предикатный символ с возможным отрицанием):
%$$\zeta := \forall x_1.\forall x_3\dots\forall x_{n-1}.
%    \bigwedge_{k=\overline{1,c}}\left(\bigvee_{i = \overline{1,d(c)}}\delta^k_i\right)$$

\end{frame}

\begin{frame}{Шаги рассуждения}
\begin{enumerate}
\item \color{gray}Упростим формулу --- поверхностные кванторы всеобщности, сколемизация, КНФ.
\item \color{black}Заменяем модель ($D$ и значения функциональных и предикатных символов).
\item \color{gray}Правильный перебор
\end{enumerate}
\end{frame}


\begin{frame}{Эрбранов универсум, основные термы}
\begin{dfn}Пусть $\varphi$ --- формула и $\mathcal{F}_k$ --- все $k$-местные функциональные символы из $\varphi$.
Тогда:\\
$H^0_\varphi := \mathcal{F}_0$ (либо $\{a\}$, если $\mathcal{F}_0 = \varnothing$);
$H^{k+1}_\varphi := H^k_\varphi \cup \{``f(``++x_1++``,``\dots``,``++x_n++``)``\ |\ x_i \in H^k_\varphi, f \in \mathcal{F}_0\}$\\

Тогда $H_\varphi = \bigcup_n H^n_\varphi$ --- эрбранов универсум, его элементы --- основные термы.
\end{dfn}

\begin{exm}[$\varphi := P(a)\vee Q(f(b))$]\vspace{-0.3cm}
$$\begin{array}{ll}
H^0_\varphi &= \{a,b\}\\
H^1_\varphi &= \{a,b,f(a),f(b)\}\\
H^2_\varphi &= \{a,b,f(a),f(b),f(f(a)),f(f(b))\}\\&\dots\\
H_\varphi &= \{f^{(n)}(x)\ |\ n \in \mathbb{N}_0, x \in \{a,b\}\}\end{array}$$
\end{exm}\vspace{-0.5cm}

\begin{exm}
\begin{tabular}{ll}
$\varphi := P(0)\vee (P(x)\rightarrow P(x'))$ & $H_\varphi = \{0, 0', 0'', 0''', \dots\}$ \\
$\varphi := P(x')$ & $H_\varphi =\{a,a',a'',a''',\dots\}$
\end{tabular}
\end{exm}
\end{frame}

\begin{frame}{Эрбранова интерпретация}


\begin{dfn}Для бескванторной $\varphi$ рассмотрим $H_\varphi$,
зададим оценку функциональных символов $f$ из $\varphi$:
$$\mathcal{F}_f(\llbracket\overline{\theta}\rrbracket) := ``f(`` ++ \llbracket \overline{\theta} \rrbracket ++ ``)``$$

Оценку для $P$ ($k$-местного предикатного символа из $\varphi$)
зададим набором истинных значений $S_P \subseteq (H_\varphi)^k$:\vspace{-0.3cm}

$$P(\theta_1,\dots,\theta_{a(i)}) \text{ истинно тогда и только тогда, когда }
\langle\llbracket\theta_1\rrbracket,\dots,\llbracket\theta_k\rrbracket\rangle \in S_P$$

Также пусть $E: \mathcal{V}\rightarrow H_\varphi$, тогда 
$\langle H_\varphi, \mathcal{F}, \mathcal{P}, E\rangle$
задаёт \emph{эрбранову интерпретацию}.
\end{dfn}

%Иными словами, для формулы

\begin{exm}
Пусть $\varphi := P(0)\vee (P(x)\rightarrow P(x'))$ и $S_P := \{0', 0'', 0'''''\}$, тогда
$$\begin{array}{l}\llbracket \varphi \rrbracket^{ x:=0 } = \llbracket P(0)\vee (P(0)\rightarrow P(0')) \rrbracket = \text{И}\\
\llbracket \varphi \rrbracket^{ x:=0'' } = \llbracket P(0)\vee (P(0'')\rightarrow P(0'''))\rrbracket = \text{Л}\end{array}$$
\end{exm}
\end{frame}

\begin{frame}{Выполнимость не теряется. Заменяем $D$ на $H$}
\begin{thm}Формула выполнима тогда и только тогда, когда она выполнима на Эрбрановом универсуме.\end{thm}
\begin{proof}
$(\Rightarrow)$ Пусть $M \models\forall \overline{x}.\varphi$. Тогда построим отображение $\text{eval}: H \rightarrow M$
(смысл названия вдохновлён языками программирования: $\text{eval}(``f(f(b))``)$ перейдёт в $f(f(b))$, где $f$ и $b$ --- из $M$).

Предикатам дадим согласованную оценку:
$P_H(t_1,\dots,t_n) = P_M(eval(t_1),\dots,eval(t_n))$. Очевидно, любая формула сохранит своё значение, кванторы всеобщности
по меньшему множеству также останутся истинными.

$(\Leftarrow)$ Очевидно.
\end{proof}\end{frame}

\begin{frame}{Шаги рассуждения}
\begin{enumerate}
\item \color{gray}Упростим формулу --- поверхностные кванторы всеобщности, сколемизация, КНФ.
\item \color{gray}Заменяем модель.
\item \color{black}Правильный перебор.
\end{enumerate}
\end{frame}

%\begin{frame}{Метод резолюции (общая схема)}
%Дана формула $\alpha$.
%\begin{enumerate}
%\item Вместо $\models\alpha$ рассматриваем невыполнимость $\neg\alpha$.
%\item По $\neg\alpha$ построим эквивалентную сколемизировнную формулу $\beta$:
%$$\beta := \forall x_1.\forall x_2.\forall x_k.\delta_1(x_1,\dots,x_k)\with\dots\with\delta_n(x_1,\dots,x_k)$$
%$\alpha$ доказуема тогда и только тогда, когда при всех оценках
%предикатных и функциональных символов (включая сколемовские функции $e_k$)
%$\beta$ невыполнима.
%\item Упрощаем предметное множество --- заменили произвольный $D$ на эрбранов универсум $H$. 
%Выполнимость формулы эквивалентна выполнимости на эрбрановом универсуме (= выполнима в какой-то эрбрановой интерпретации).
%
%\item Осталось избавиться от кванторов всеобщности и организовать правильный перебор
%($H$ может быть бесконечным).
%\end{enumerate}
%\end{frame}

\begin{frame}{Противоречивые системы дизъюнктов}
%\begin{thm}[о выполнимости]Формула выполнима тогда и только тогда, когда она выполнима в какой-то эрбрановой оценке.\end{thm}
%\begin{proof}Доказано на предыдущей лекции.\end{proof}

\begin{dfn}Система дизъюнктов $S = \{\delta_1,\dots,\delta_n\}$ противоречива,
если для каждой оценки $M = \langle D,P,F,E \rangle$ найдётся $\delta_t$ и такой набор $\overline{d} \in D$,
что $\llbracket\delta_t\rrbracket^{\overline{x} := \overline{d}} = \text{Л}$.
\end{dfn}

\begin{thm}Система дизъюнктов противоречива, если она невыполнима в эрбрановых интерпретациях.\end{thm}
\end{frame}

\begin{frame}{Основные примеры.}

Рассмотрим сколемизированную формулу $\beta$ в КНФ. Заметим, что если $\beta = \forall x_1.\dots.\forall x_k.\delta_1\with\delta_2\with\dots\with\delta_n$,
то $$\vdash \beta \leftrightarrow (\forall x_1.\dots.\forall x_k.\delta_1)\with\dots\with(\forall x_1.\dots.\forall x_k.\delta_n)$$

\begin{dfn}
Дизъюнкт с подставленными значениями из эрбранового универсума $H_\beta$ (как строками) вместо переменных называется основным примером формулы $\beta$.
\end{dfn}
\begin{exm}Пусть $\beta := \forall x.P(0) \with (P(x)\vee P(x'))$, тогда $P(0''')\vee P(0'''')$ --- основной пример, а $P(0''''')$ --- нет.
\end{exm}


\end{frame}

\begin{frame}{Противоречивые множества основных примеров}

\begin{dfn}Система основных примеров --- все основные примеры, опровергаемые хоть при какой-то эрбрановой интерпретации $\mathcal{M}$:

\vspace{-0.3cm}
$$\mathcal{E}_S = \{ \delta_t[\overline{x} := \overline{d}]\ |\ \text{существует }\mathcal{M}\text{, что } \llbracket \delta_t \rrbracket^{\overline{x} := \overline{d}}_\mathcal{M}=\text{Л};\quad d_i \in H_\beta\}$$
\end{dfn}

\begin{dfn}Система основных примеров $E$ противоречива в эрбрановых интерпретациях, если 
для любой эрбрановой интерпретации $\mathcal{M}$ найдётся такой $\varepsilon\in E$, 
что $\llbracket \varepsilon \rrbracket_\mathcal{M} = \text{Л}$.
\end{dfn}

%\end{dfn}\vspace{-0.5cm}
\begin{thm}Система дизъюнктов $S$ противоречива тогда и только тогда, когда система её 
основных примеров $\mathcal{E}_S$ противоречива в эрбрановых интерпретациях.\end{thm}
%\begin{proof}Для некоторой эрбрановой интерпретации дизъюнкт $\delta_k$
%опровергается тогда и только тогда, когда соответствующая ему подстановка в $\mathcal{E}_S$ опровергается.
%\end{proof}
\end{frame}

\begin{frame}{Теорема Эрбрана}
\begin{thm}[Гёделя о компактности]Если $\Gamma$ --- некоторое семейство бескванторных формул, то $\Gamma$ имеет модель
тогда и только тогда, когда любое его конечное подмножество имеет модель.\end{thm}

\begin{thm}[Эрбрана]Система дизъюнктов $S$ противоречива тогда и только тогда, когда у
$\mathcal{E}_S$ существует конечное противоречивое в эрбрановой интерпретации подмножество.\end{thm}
\begin{proof}$(\Leftarrow)$ 
Пусть $\{\varepsilon_1,\dots,\varepsilon_t\} \subseteq \mathcal{E}_S$ противоречиво, $\varepsilon_i = \delta_{m_i}[\overline{x} := \overline{d_i}]$,
где $\overline{d_i}$ --- набор значений из $H$. 
То есть, для любой эрбрановой интерпретации $M$ существует $\varepsilon_p$, что $\llbracket\varepsilon_p\rrbracket_M = \text{Л}$. 
%Тогда $\llbracket \delta_{m_p} \rrbracket^{\overline{x}:=\overline{d_p}}_M = \text{Л}$.
Отсюда, по теореме о выполнимости $S$ тоже противоречива.

$(\Rightarrow)$ Если $S$ противоречива, то $\mathcal{E}_S$ противоречива. 
Тогда у неё нет модели. Тогда у неё найдётся конечное противоречивое подмножество (компактность).
\end{proof}

Возможно убедиться в невыполнимости за конечное время.
\end{frame}

\begin{frame}{Главное --- не запутаться в определениях}
\begin{itemize}
\item Показываем невыполнимость формулы $\varphi = \bigvee\bigwedge\delta_i$ (в КНФ).
\item По $\varphi$ строим $H_\varphi$ (эрбранов универсум, состоит из основных термов)
\item Доопределяем функциональные символы как конкатенацию строк (эрбранова интерпретация).
Выполнимость формулы эквивалентна её выполнимости в эрбрановой интерпретации.
\item Заменяем формулу $\varphi$ на множество $\{\delta_1,\dots,\delta_n\}$ (система дизъюнктов $S$, убираем кванторы)
\item Рассматриваем систему дизъюнктов с подставленными значениями из $H_\varphi$ (основные примеры, убираем переменные).
\item Оставляем только полезные --- те, что опровергаются хотя бы в какой-то эрбрановой интерпретации ($\mathcal{E}_S$, система основных примеров).
\item Невыполнимость формулы эквивалентна невыполнимости (про\-ти\-во\-ре\-чи\-вос\-ти) системы дизъюнктов и эквивалентна
противоречивости системы основных примеров.
\end{itemize}
\end{frame}

\begin{frame}{Общая схема алгоритма}
Цель алгоритма: убедиться, что $\alpha$ доказуемо.
\begin{enumerate}
\item По формуле $\alpha$ строим её отрицание $\neg\alpha$.
\item Приводим к виду с поверхностными кванторами, проводим сколемизацию, находим КНФ:
$\beta = \forall x_1.\dots.\forall x_k.\delta_1\with\dots\with\delta_n$.
\item Убедимся, что при любом $D$ и значениях функциональных и предикатных символов и сколемовских функций $e_k$ найдутся $d_i \in D$, 
что один из дизъюнктов $\delta_t$ при подстановке $\overline{x} := \overline{d}$ ложный.
\item Для этого строим универсум Эрбрана $H$, и систему основных примеров $\mathcal{E}_S$, её противоречивость эквивалентна невыполнимости $\beta$.
\item Конечное противоречивое подмножество по теореме Эрбрана обязательно находится в каком-то начальном 
отрезке $\{\varepsilon_1,\dots,\varepsilon_t\} \subseteq \mathcal{E}_S$ 
(если оно есть).
\end{enumerate}
\end{frame}

\begin{frame}{Пример: как проверяем выполнимость формулы?}
Допустим, формула: $(\forall x.P(x)\with P(x')) \with \exists x.\neg P(x'''')$

\begin{enumerate}
\item Поверхностные кванторы, сколемизация, КНФ: $(\forall x.P(x)) \with (\forall x.P(x')) \with (\neg P(e''''))$
\item Строим эрбранов универсум: $H = \{e, e', e'', e''', \dots \}$
\item Если есть противоречие, то среди основных примеров:
$$\mathcal{E} = \{ P(e), P(e'), P(e''), P(e'''), P(e''''), \neg P(e''''), \dots \}$$
\end{enumerate}

Либо есть $\mathcal{M}$, что $\llbracket\bigwith \mathcal{E}\rrbracket_\mathcal{M} = \text{И}$, 
либо есть $\{\varepsilon_1,\dots,\varepsilon_n\} \subseteq \mathcal{E}$, что $\llbracket\varepsilon_t\rrbracket_\mathcal{M} = \text{Л}$
для какого-то $t$ при каждой эрбрановой интерпретации $\mathcal{M}$.
\vspace{0.3cm}

\begin{tabular}{lll}
Подмножество $\mathcal{E}$& выполнено в интерпретации & количество интерпретаций\\\hline
$\{ P(e) \}$ & $\llbracket P(e) \rrbracket  = \text{И}$ & 2 варианта\\
$\{ P(e), P(e') \}$ & $\llbracket P(e) \rrbracket = \llbracket P(e') \rrbracket  = \text{И}$ & 4 варианта\\
\dots\\
$\{ P(e), \dots, P(e''''), \neg P(e'''') \}$ & невыполнимо & 32 варианта
\end{tabular}
\end{frame}

%Добавляем по примеру и проверяем.
%$P(e)$ система выполнима при $\llbracket P(e) \rrbracket  = \text{И}$ (2 варианта)
%$P(e')$ система выполнима при $\llbracket P(e') \rrbracket = \text{И}$ (4 варианта)
%...
%$P(e'''')$ система выполнима при $\llbracket P(e'''') \rrbracket = \text{И}$ (32 варианта)
%$\neg P(e'''')$ система невыполнима (64 варианта).

\begin{frame}{Правило резолюции (исчисление высказываний)}
Пусть даны два дизъюнкта, $\alpha_1 \vee \beta$ и $\alpha_2 \vee \neg \beta$.
Тогда следующее правило вывода называется правилом резолюции:

$$\infer{\alpha_1\vee \alpha_2}{\alpha_1\vee \beta\quad\quad \alpha_2\vee\neg \beta}$$

\begin{thm}Система дизъюнктов противоречива, если в процессе всевозможного применения
правила резолюции будет построено явное противоречие,
т.е. найдено два противоречивых дизъюнкта: $\beta$ и $\neg\beta$.
\end{thm}
\end{frame}

\begin{frame}{Расширение правила резолюции на исчисление предикатов}
Заметим, что правило резолюции для исчисления высказываний не подойдёт для исчисления предикатов.

$$S = \{ P(x), \neg P(0)\}$$

Здесь $P(x)$ противоречит $\neg P(0)$, но правило резолюции для исчисления высказываний здесь неприменимо, потому
что $x$ можно заменять, это не константа:
%$\beta \equiv P(x)$, тогда $\neg \beta \not\equiv \neg P(0)$.

$$\infer{???}{P({\color{red}x})\quad\quad\neg P({\color{red}0})}$$

Нужно заменять $P(x)$ на основные примеры, и искать среди них. Модифицируем правило резолюции для этого.

\end{frame}


\begin{frame}{Алгебраические термы}
	\begin{dfn}Алгебраический терм $$\theta := x\:|\:(f(\theta_1,\ldots,\theta_n))$$ 
где $x-$переменная, $f(\theta_1,\ldots,\theta_n)-$применение функции. Напомним, что константы --- нульместные
функциональные символы, собственно переменные будем обозначать последними буквами латинского алфавита. \end{dfn}
	%\subsection{Уравнение в алгебраических термах $\theta_1=\theta_2$\\Система уравнений в алгебраических термах}
	\begin{dfn}Система уравнений в алгебраических термах
	$
		\begin{cases}
			\theta_1=\sigma_1&\\
			\vdots&\\
			\theta_n=\sigma_n&\\
		\end{cases}
	$\par где $\theta_i \text{ и } \sigma_i-\text{термы}$\par
\end{dfn}
\end{frame}
\begin{frame}{Уравнение в алгебраических термах}
	\begin{dfn}$\{x_i\}=X-$множество переменных, $\{\theta_i\}=T-$множество термов.\end{dfn}
	\begin{dfn}Подстановка$-$отображение вида: $\pi_0:X\to T$, тождественное почти везде (за исключением конечного числа переменных).
        \par $\pi_0(x)$ может быть либо $\pi_0(x)=\theta_i\text{, либо }\pi_0(x)=x$.\end{dfn} 
	Доопределим $\pi:T\to T$, где \begin{enumerate}
		\item $\pi(x)=\pi_0(x)$
		\item $\pi(f(\theta_1, \ldots, \theta_k))=f(\pi(\theta_1), \ldots, \pi(\theta_k))$
	\end{enumerate}
	
	\begin{dfn}Решить уравнение в алгебраических термах$-$найти такую наиболее общую подстановку $\pi$, 
        что $\pi(\theta_1)=\pi(\theta_2)$.
Наиболее общая подстановка --- такая, для которой другие подстановки являются её частными случаями.\end{dfn} 
\end{frame}

\begin{frame}{Задача унификации}
\begin{dfn}
Пусть даны формулы $\alpha$ и $\beta$. Тогда решением задачи унификации
будет такая наиболее общая подстановка $\pi = \mathcal{U}\big[\alpha,\beta\big]$, что $\pi(\alpha) = \pi(\beta)$.

%Будем писать $\Sigma = \mathcal{U}\[\alpha,\beta\]$, если $\Sigma(\alpha) = \Sigma(\beta) = \eta$ и $\Sigma$ --- наиболее общая.
Также, $\pi$ назовём наиболее общим унификатором.
\end{dfn}

\begin{exm}
\begin{itemize}
%\item $\mathcal{U}\[ P(x,g(b)),P(f(a),y) \] = [ x := f(a), y := f(b) ]$ 
%и $P(f(a),g(b))$ --- унификатор.
\item Формулы $P(a,g(b))$ и $P(c,d)$ не имеют унификатора (мы считаем, что $a,b,c,d$ --- нульместные функции, а
$g$ --- одноместная функция).

\item Проверим формулу на соответствие 11 схеме аксиом $(\forall x.\varphi)\rightarrow\varphi[x := \theta]$: $$(\forall x.P(x))\rightarrow P(f(t,g(t),y))$$
Для этого решим задачу унификации: $\pi = \mathcal{U}\big[P(x),P(f(t,g(t),y))\big]$, тогда $\pi(x) = f(t,g(t),y)$.
\end{itemize}
\end{exm}
\end{frame}

\begin{frame}{Правило резолюции для исчисления предикатов}
\begin{dfn}
Пусть $\sigma_1$ и $\sigma_2$ --- подстановки, заменяющие переменные в формуле на свежие. 
Тогда правило резолюции выглядит так:

$$\infer[{\pi = \mathcal{U}\big[\sigma_1(\beta_1),\sigma_2(\beta_2)\big]}]
        {\pi(\sigma_1(\alpha_1)\vee \sigma_2(\alpha_2))}
        {\alpha_1\vee \beta_1\quad\quad\alpha_2\vee\neg \beta_2}$$
\end{dfn}

$\sigma_1$ и $\sigma_2$ разделяют переменные у дизъюнктов, чтобы $\pi$ не осуществила лишние
замены, ведь $\vdash(\forall x.P(x) \with Q(x)) \leftrightarrow (\forall x.P(x))\with(\forall x.Q(x))$, но
$\not\vdash (\forall x.P(x) \vee Q(x)) \rightarrow (\forall x.P(x))\vee(\forall x.Q(x))$.

\begin{exm}\vspace{-0.5cm}
$$\infer[\text{ подстановки: } \sigma_1(x) = x', \sigma_2(x) = x'', \pi(x')=a]{Q(a)\vee T(x'')}{Q(x)\vee P(x) \quad \neg P(a)\vee T(x)}$$
\end{exm}
\end{frame}

\begin{frame}{Метод резолюции}
Ищем $\vdash\alpha$.

\begin{enumerate}
\item будем искать опровержение $\neg\alpha$.
\item перестроим $\neg\alpha$ в КНФ.
\item будем применять правило резолюции, пока получаем новые дизъюнкты и пока 
не найдём явное противоречие (дизъюнкты вида $\beta$ и $\neg\beta$).
\end{enumerate}

Если противоречие нашлось, значит, $\vdash\neg\neg\alpha$. Если нет --- значит, $\vdash\neg\alpha$.
Процесс может не закончиться.
\end{frame}

\begin{frame}{SMT-решатели}

Обычно требуется не логическое исчисление само по себе, а теория первого порядка.
То есть, <<Satisfiability Modulo Theory>>, <<выполнимость в теории>> --- вместо SAT, выполнимости.
\begin{itemize}
\item Иногда можно вложить теорию в логическое исчисление, 
даже в исчисление высказываний: $\overline{S_2S_1S_0} = \overline{A_1A_0}+\overline{B_1B_0}$
$$\begin{array}{ll}
S_0 = A_0 \oplus B_0 & C_0 = A_0 \with B_0\\
S_1 = A_1 \oplus B_1 \oplus C_0 & C_1 = (A_1 \with B_1) \vee (A_1 \with C_0) \vee (B_1 \with C_0) \\
S_2 = C_1\end{array}$$

\item А можно что-то добавить прямо на уровень унификации / резолюции:
Например, можем зафиксировать арифметические функции --- и производить вычисления
в правиле резолюции вместе с унификацией.

Тогда противоречие в $\{x = 1+3+1,\neg x = 5\}$ можно найти за один шаг.
\end{itemize}
\end{frame}

\begin{frame}[fragile]{Уточнённые типы (Refinement types), LiquidHaskell}
\begin{dfn}(Неформальное) Уточнённый тип --- тип вида $\{\tau(x)\ |\ P(x)\}$, где $P$ --- некоторый предикат.\end{dfn}

Пример на LiquidHaskell:
\begin{verbatim}
data [a] <p :: a -> a -> Prop> where
   | []  :: [a] <p>
   | (:) :: h:a -> [a<p h>]<p> -> [a]<p>
\end{verbatim}
\begin{itemize}
\item \verb!h:a! --- голова ($h$) имеет тип $a$\\
\item \verb![a<p h>]<p>! --- хвост состоит из значений типа $a$, уточнённых $p$ --- $\{ t : a\ |\ p\ h\ t\}$ (карринг: \verb!a <p h>!).
\end{itemize}

\begin{verbatim}
{-@ type IncrList a = [a] <{\xi xj -> xi <= xj}> @-}
{-@ insertSort    :: (Ord a) => xs:[a] -> (IncrList a) @-}
insertSort []     = []
insertSort (x:xs) = insert x (insertSort xs) 
\end{verbatim}
\end{frame}

\end{document}
