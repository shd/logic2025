\documentclass[11pt,a4paper,oneside]{scrartcl}
\usepackage[utf8]{inputenc}
\usepackage[english,russian]{babel}
\usepackage[top=1cm,bottom=1cm,left=1cm,right=1cm]{geometry}

\begin{document}
\pagestyle{empty}

\begin{center}
{\large\scshape\bfseries Программа курса <<Математическая логика>>}\\
{\large\scshape Вопросы ко второму коллоквиуму.}\\
\itshape ИТМО, группы M3232--M3239, осень 2025 г.
\end{center}

%\vspace{0.3cm}

\begin{enumerate}
\item Исчисление предикатов.
Теория моделей исчисления предикатов (что такое оценка).
Общезначимость, следование.
Вхождения, свободные вхождения, подстановка, свобода для подстановки.
Теория доказательств, выводимость.
Теорема о дедукции в исчислении предикатов. Отличия от исчисления высказываний.
Лемма о перестановке подстановки и оценки. Теорема о корректности исчисления предикатов.
\item Теорема Гёделя о полноте исчисления предикатов.
Непротиворечивые множества формул (с кванторами и бескванторами).
Пополнение множества формул.
Существование моделей у непротиворечивых множеств формул в бескванторном исчислении предикатов.
Поверхностные кванторы (предварённая форма). Эквивалентность формул формулам с поверхностными кванторами.
Сколемизация.
Теорема Гёделя о полноте исчисления предикатов. 
Полнота исчисления предикатов.
Теорема Гёделя о компактности.
\item Машина Тьюринга. Разрешимость теории, примеры. Задача об останове, её неразрешимость. 
Неразрешимость исчисления предикатов.
\item Представление чисел через натуральные (целые, рациональные, вещественные). 
Аксиоматика Пеано. Арифметические операции (сложение, умножение) в аксиоматике Пеано.
\item Порядок теории (0, 1, 2). Теории первого порядка. Формальная арифметика. 
\item Примитивно-рекурсивные и рекурсивные функции. 
Функции вычисления простых чисел. Частичный логарифм.
Выразимость отношений и представимость функций в формальной арифметике. Характеристические функции.
Функция Аккермана. 
\item Бета-функция Гёделя. 
Гёделева нумерация. Рекурсивность представимых в формальной арифметике функций.
\item Непротиворечивость (эквивалентные определения), $\omega$-не\-про\-ти\-во\-ре\-чи\-вость. 
Первая теорема Гёделя о неполноте арифметики.
Формулировка первой теоремы Гёделя о неполноте арифметики в форме Россера. 
Синтаксическая и семантическая неполнота арифметики.
Неполнота расширений формальной арифметики.
Ослабленные варианты: арифметика Пресбургера, система Робинсона.
\item Вторая теорема Гёделя о неполноте арифметики, $Consis$. 
Лемма об автоссылках. Условия Гильберта-Бернайса-Лёба. Неразрешимость формальной арифметики. 
Теорема Тарского о невыразимости истины.
\end{enumerate}
\end{document}
