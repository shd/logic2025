\documentclass[aspectratio=169]{beamer}
\usepackage[utf8]{inputenc}
\usepackage[english,russian]{babel}
\usepackage{cancel}
\usepackage{amssymb}
\usepackage{stmaryrd}
\usepackage{cmll}
\usepackage{graphicx}
\usepackage{amsthm}
\usepackage{tikz}
\usepackage{multicol}
\usetikzlibrary{patterns,calc}
\usepackage{chronosys}
\usepackage{proof}
\usepackage{multirow}
\usepackage{marvosym}
\usepackage{hyperref}
\usepackage{comment}
\setbeamertemplate{navigation symbols}{}
%\usetheme{Warsaw}

\newtheorem{thm}{Теорема}[section]
\newtheorem{dfn}{Определение}[section]
\newtheorem{lmm}{Лемма}[section]
\newtheorem{exm}{Пример}[section]
\newtheorem{snote}{Пояснение}[section]

\newcommand{\divisible}%
{\mathrel{\lower.2ex%
\vbox{\baselineskip=0.7ex\lineskiplimit=0pt%
\kern6pt \hbox{.}\hbox{.}\hbox{.}}%
}}

\begin{document}

\newcommand\doubleplus{+\kern-1.3ex+\kern0.8ex}
\newcommand\mdoubleplus{\ensuremath{\mathbin{+\mkern-10mu+}}}

\begin{frame}{Ординалы (порядковые числа) и порядок}
\begin{exm}\begin{itemize}
\item Добавить элемент перед бесконечностью: $\mathbb{N}$ и $\mathbb{N}_0$.
\pause
$1 + \omega = \omega$. \pause
\item Добавить элемент после бесконечности $(+\infty)$. \pause $\omega + 1 \ne \omega$ \pause
%\includegraphics[scale=0.9]{pics/lection-13-ghc}
\end{itemize}\end{exm}
\end{frame}

\begin{frame}{Пары и списки}
\begin{exm}Упорядоченные пары натуральных чисел имеют порядковый тип $\omega^2$.\pause

\begin{center}$\langle 3,5 \rangle < \langle 4,3 \rangle\quad\quad\omega \cdot 3 + 5 < \omega \cdot 4 + 3$.\end{center}\end{exm}\pause

\begin{exm}Списки натуральных чисел --- порядковый тип $\omega^\omega$.
$$\langle 3,1,4,1,5,9\rangle\quad\quad \omega^5 \cdot 3 + \omega^4 \cdot 1 + \omega^3 \cdot 4 + \omega^2 \cdot 1 + \omega^1 \cdot 5 + 9$$\end{exm}
\end{frame}

\begin{frame}{Дизъюнктные множества}
\begin{dfn}Дизъюнктное (разделённое) множество --- множество, элементы которого
не пересекаются. 
$$Dj(x) \equiv \forall y.\forall z.(y \in x \with z \in x \with \neg y=z) \rightarrow 
\neg \exists t.t \in y \with t \in z$$
\end{dfn}\pause

\begin{exm}Дизъюнктное: $\{\{1,2\},\{\rightarrow\},\{\alpha,\beta,\gamma\}\}$\\ \pause
Не дизъюнктное: $\{\{1,2\},\{\rightarrow\},\{\alpha,\beta,\gamma,1\}\}$
\end{exm}
\end{frame}

\begin{frame}{Прямое произведение множеств}
\begin{dfn}Прямое произведение дизъюнктного множества $a$ --- 
множество $\times a$ всех таких множеств $b$, что:
\begin{itemize}
\item $b$ пересекается с каждым из элементов множества $a$ в точности в одном элементе
\item $b$ содержит элементы только из $\cup a$.
\end{itemize}

$$\forall b .b \in \times a \leftrightarrow (b \subseteq \cup a \with \forall y .y \in a \rightarrow \exists ! x .x \in y \with x \in b)$$
\end{dfn}\pause

\begin{exm}
$\times\{\{\triangle,\square\},\{1,2,3\}\} = \{\{\triangle,1\},\{\triangle,2\},\{\triangle,3\},\{\square,1\},\{\square,2\},\{\square,3\}\}$
\end{exm}

\end{frame}

\begin{frame}{Аксиома выбора}
\begin{dfn}
Прямое произведение непустого дизъюнктного множества, 
не содержащего пустых элементов, непусто.

$$\forall t.Dj (t) \rightarrow 
(\forall x.x \in t \rightarrow \exists p.p \in x) \rightarrow
(\exists p.p \in \times t)$$
\end{dfn}\pause

Альтернативные варианты: любое множество можно вполне упорядочить, \pause любая сюръективная функция имеет частичную обратную, 
и т.п.
\begin{dfn}Аксиоматика ZF + аксиома выбора = ZFC\end{dfn}\pause
\end{frame}

\begin{frame}{Дискуссия вокруг аксиомы выбора}
\begin{exm}Парадокс Банаха-Тарского: трёхмерный шар равносоставен двум своим копиям.\end{exm}\pause
\begin{thm}Теорема (Гёдель, 1938): аксиома выбора не добавляет противоречий в ZF.\end{thm}\pause
\begin{thm}Теорема (Коэн, 1963): аксиома выбора не следует из других аксиом ZF.\end{thm}\pause
\begin{exm}Односторонние функции: Sha256 и т.п. У Sha256 практически невозможно найти обратную.\end{exm}\pause
\begin{thm}Теорема Диаконеску: ZFC поверх интуиционистского исчисления предикатов содержит правило исключённого третьего.\end{thm}
\end{frame}

\begin{frame}{Аксиома фундирования}
\begin{dfn}Аксиома фундирования. 
В каждом непустом множестве найдётся элемент, не пересекающийся с исходным множеством.
$$\forall x .x = \varnothing \vee \exists y .y \in x \with \forall z.z \in x \rightarrow z \notin y$$
\end{dfn}

Иными словами, в каждом множестве есть элемент, минимальный по отношению $(\in)$.

Идея Рассела: каждому множеству припишем \emph{тип} (тип пустого 0, тип множеств 1,
тип множеств множеств 2 и т.п.). Тогда конструкция невозможна: $\{ x\ |\ x \in x\}$.
Аксиома фундирования позволяет определить функцию ранга:
$$rk(x) = \text{upb }\{rk(y)\ |\ y\in x\}$$.
\end{frame}

\begin{frame}{Схема аксиом подстановки}
\begin{dfn}Схема аксиом подстановки. 
Пусть задана некоторая функция f, представимая в исчислении предикатов:
то есть задана некоторая формула $\phi$, такая, что $f(x) = y$
тогда и только тогда, когда $\phi(x,y) \with \exists ! z. \phi(x,z)$.
Тогда для любого множества S существует множество f(S) --- образ
множества S при отображении f.
$$\forall s .(\forall x .\forall y_1 .\forall y_2 .x \in s \with \phi (x,y_1) \with \phi
(x,y_2) \rightarrow y_1=y_2) \rightarrow 
(\exists t .\forall y .y \in t
\leftrightarrow \exists x . x \in s \with \phi (x,y)) $$
\end{dfn}
\end{frame}

\end{document}
