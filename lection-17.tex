\documentclass[aspectratio=169,dvipsnames,usenames]{beamer}
\usepackage{graphicx}
\usepackage{stmaryrd}
\usepackage[T1,T2A]{fontenc}
\usepackage[utf8]{inputenc}
\usepackage[english,russian]{babel}
\usepackage{amsmath}
\usepackage{amsfonts}
\usepackage{amssymb}
\usepackage{makeidx}
\usepackage{verbatim}
\usepackage{amsthm}
%\usepackage{bnf}
\usepackage{cmll}
\usepackage{tikz}
\usepackage{enumerate}
\usepackage{mathtext}
\usepackage{mathtools}
\usepackage{mathabx}
%\usepackage[left=2cm,right=2cm,top=2cm,bottom=2cm,bindingoffset=0cm]{geometry}
\usepackage{proof}
%\usepackage{paracol}
%\usepackage{enumitem}
\usepackage{xcolor}
\usepackage{colortbl}
%\usepackage{minted}
%\usepackage{hyperref}
\setbeamertemplate{navigation symbols}{}
\usetikzlibrary{graphs}
\usetikzlibrary{graphs.standard}
\usetikzlibrary{automata,positioning}
\usepackage{float}

\begin{document}

%\theoremstyle{dfn}
\newtheorem{dfn}{Определение}[section]
\newtheorem{nte}{Замечание}[section]

\newtheorem{axiom}{Аксиома}[section]
\newtheorem{thm}{Теорема}[section]
\newtheorem{lmm}[theorem]{Лемма}
\newtheorem{statement}{Утверждение}[section]
\newtheorem{oun_paragraph}{Пункт}[section]
\newtheorem{cons}{Следствие}[section]
\newtheorem*{exm}{Пример}

\newcommand{\comb}[1]{\operatorname{\mathcal{#1}}}
\newcommand{\func}[1]{\operatorname{#1}}
\newcommand{\reduction}[1]{{\color{OrangeRed}#1}}
\newcommand{\set}[1]{\left\{#1\right\}}

\def\from#1{\par \parbox{0.7\textwidth}{\par \hfill\raggedleft \it #1}} 

\begin{frame}{}
\begin{center}
{\LARGE Линейная темпоральная логика\\Проверка свойств на моделях (model checking)}
\end{center}
\end{frame}

\begin{frame}{Модальные логики}
\begin{itemize}
\item Модальность (лат. modus — способ, вид) — способ, вид бытия или события; 
категории модальности: возможность, действительность, необходимость.

\item Расширяем язык: как бы выразить <<модальности>>?
\emph{Всегда} зимой идёт снег. Дождь \emph{может} идти при солнечном свете.

\item Модифицируем язык, модифицируем аксиоматику, модифицируем теорию моделей.
\item Язык предполагает включение новых связок, самые типичные:
\begin{tabular}{ll}
$\Box$ & необходимость (necessity)\\
$\Diamond$ & возможность (possibility)
\end{tabular}

\item Интуитивный смысл связок примерно понятен, конкретный смысл формализуется
в конкретной теории.

\end{itemize}
\end{frame}

\begin{frame}{Некоторые модальные исчисления (обзор)}
Терминология введена Кларенсом Льюисом и Купером Лангфордом в 1932 году.
\begin{itemize}
\item Минимальная модальная логика (K) строится поверх ИВ:
$$\text{Аксиомы ИВ}\quad\quad\Box(\varphi\rightarrow\psi)\rightarrow(\Box\varphi\rightarrow\Box\psi)
\quad\quad\infer{\psi}{\varphi\quad\varphi\rightarrow\psi}\quad\quad\infer{\Box\varphi}{\varphi}$$

\item K4: K и дополнительная аксиома транзитивности  $\Box\varphi \rightarrow \Box\Box\varphi$;
\item T: K и дополнительная аксиома рефлексивности $\Box\varphi \rightarrow \varphi$;
\item S4: K4 + T;
\item S5: T и аксиома $\Diamond\varphi\rightarrow\Box\Diamond\varphi$
(<<если что-то возможно, то оно обязательно реализуется в каком-то мире>>).
Например, утверждение <<если Бог возможен, то он необходимо существует>>
можно формализовать и доказать в S5.
\end{itemize}
\end{frame}

\begin{frame}{Теория моделей}
\begin{itemize}
\item Много разных --- например, топологические интерпретации (как и в ИИВ).
\item Наш интерес сегодня --- интерпретации на многих мирах (Саул Крипке), где
миры упорядочены некоторым отношением. Тогда интуитивный смысл связок:
$\Box$ --- истинно во всех достижимых мирах, $\Diamond$ --- истинно
в каком-то достижимом мире. Можем указывать конкретный мир $W$:

$$W \models \Box A$$

\item Скажем, следующее выполнено в интуиционистских моделях: %$W\Vdash\varphi$, если...

$$W\models X\rightarrow\Box X\quad\quad W\models(\alpha\rightarrow\beta) \rightarrow (\Box (\alpha\rightarrow\beta))
\quad\quad W\models\neg\alpha\rightarrow\neg\Diamond\alpha$$

\end{itemize}
\end{frame}

\begin{frame}{Линейная темпоральная логика}
\begin{itemize}
\item Темпоральная логика: множественные миры (в стиле моделей Крипке) понимаются как 
расположенные в соответствии с течением времени.
\item Линейная темпоральная логика: миры выстроены в линейном порядке.
\item Используем следующие связки:
\begin{enumerate}
\item $\mathcal{G}(\alpha)$ или $\Box\alpha$: утверждение $\alpha$ выполнено в любой момент (начиная с текущего).
\item $\mathcal{P}(\alpha)$ или $\bigcirc\alpha$: утверждение $\alpha$ выполнено в следующий момент.
\item $\mathcal{E}(\alpha)$ или $\Diamond\alpha$: утверждение $\alpha$ неизбежно выполнено в будущем, в какой-то момент (начиная с текущего).
\item $\mathcal{U}(\alpha,\beta)$ или $\alpha\mathcal{U}\beta$: утверждение $\alpha$ истинно, пока $\beta$ не станет истинным,
после чего $\alpha$ может быть любым.
\end{enumerate}
\end{itemize}
\end{frame}

\begin{frame}{Представление моделей ЛТЛ как множества слов}

$$W(\varphi) = \{\sigma\in (\mathcal{P}(a))^\omega\ |\ \sigma\models\varphi \}$$
На строке $\sigma = S_0S_1S_2\dots$ (каждый $S_i \subseteq \mathcal{P}(a)$) истинность задаётся так:
$$\begin{array}{ll}
\sigma\models\top & \text{всегда}\\
\sigma\models a & a \in S_0\\
\sigma\models \varphi_1 \with \varphi_2 & \sigma\models\varphi_1 \text{ и } \sigma\models\varphi_2\\
\sigma\models\neg\varphi & \sigma\not\models\varphi\\
\sigma\models\bigcirc\varphi & \sigma[1\dots]\models\varphi\\
\sigma\models\varphi_1\mathcal{U}\varphi_2 & \text{существует }j.\sigma[j\dots]\models\varphi_2 \text{ и при всех }i, 0 \le i < j. \sigma[i\dots]\models\varphi_1
\end{array}$$
\end{frame}

\begin{frame}{Выразимость связок, другие формулы}
Будем рассматривать следующую грамматику для формул: 
$$\varphi ::= \top\ |\ a\ |\ \varphi\with\varphi\ |\ \neg\varphi\ |\ \bigcirc\varphi\ |\ \varphi\mathcal{U}\varphi$$
поскольку остальные связки выражаются через эти. 

В самом деле, имеем следующие тождества:
\begin{itemize}
\item Связки выражаются друг через друга:
\begin{itemize}
\item $\Box\alpha = \neg\Diamond\neg\alpha$
\item $\Diamond\alpha = \top \mathcal{U} \alpha$
\end{itemize}
\item Правила двойственности:
\begin{itemize}
\item $\neg\bigcirc\varphi=\bigcirc\neg\varphi$
\item $\neg\Diamond\varphi = \Box\neg\varphi$
\item $\neg\Box\varphi = \Diamond\neg\varphi$
\end{itemize}
\item Правила расширения:
\begin{itemize}
\item $\varphi\mathcal{U}\psi = \psi \vee (\varphi \with \bigcirc (\varphi \mathcal{U} \psi))$
\item $\Diamond \varphi = \varphi \vee \bigcirc \Diamond \varphi$
\item $\Box \varphi = \varphi \with \bigcirc \Box \varphi$
\end{itemize}
\end{itemize}
\end{frame}

\begin{frame}{Задача проверки программы на моделях}
\begin{itemize}
\item Проверить многопоточный алгоритм/протокол --- как?
\item Первоначальная идея: Сlarke E., Emerson E., Sistla A., Automatic verification of finite-state 
concurrent systems using temporal logic specifications 
1986.
\item Опишем состояние некоторым набором (булевских) переменных.
Программа задаёт набор допустимых переходов между состояниями.
\item Запишем условие (которое желаем проверить) как формулу ЛТЛ.
\item Научимся проверять выполнимость формулы на данном множестве состояний.
\end{itemize}
\end{frame}

\begin{frame}{Постановка задачи}
\begin{dfn}Системой переходов назовём граф состояний, в котором каждое состояние отражает
содержимое памяти компьютера, а переходы соответствуют инструкциям (операциям), выполняемым
компьютером\end{dfn}

Хотим научиться проверять, выполнено ли $\varphi$ при всех возможных вариантах выполнения программы,
то есть при всех возможных путях в системе переходов $TS$:

$$TS \models \varphi$$

Очевидно, $TS$ задаёт некоторое множество (бесконечных) строк в алфавите $2^{FV(\varphi)}$.
Находится ли в этом множестве строка, удовлетворяющая $\varphi$?
\end{frame}

\begin{frame}{Недетерменированные (обобщённые) автоматы Бюхи}
\begin{dfn}
НАБ (НОАБ) определяется внешним алфавитом $A$, множествами состояний $Q$, функцией переходов 
$\delta: A\times Q \rightarrow \mathcal{P}(Q)$ и семейством допускающих множеств состояний
$\mathcal{F} \subseteq \mathcal{P}(Q)$.

Бесконечная строка $\alpha = a_0a_1a_2\dots$ допускается недетерменированным (обобщённым) автоматом Бюхи, 
если найдётся такая последовательность состояний $q_0q_1q_2\dots$, что
$q_{n+1}\in\delta(a_n,q_n)$ и в процессе применения автомата к ней каждое из множеств допускающих 
состояний будет 
посещено бесконечное количество раз.


\vspace{-0.5cm}
$$\forall F\in\mathcal{F}.\forall n\in\mathbb{N}.\exists m > n.q_m\in F$$
\vspace{-0.5cm}

В случае $|\mathcal{F}| \le 1$ такой автомат --- НАБ, иначе --- НОАБ.
\end{dfn}
\end{frame}

\begin{frame}{Пример обобщённого недетерменированного автомата Бюхи}
\begin{exm}
Рассмотрим автомат с двумя состояниями (0,1), начальным состоянием 0, и $\mathcal{F} = \{\{0\},\{1\}\}$.

\begin{tikzpicture}[->,shorten >=1pt,auto,node distance=2.8cm,
                    semithick]
  \tikzstyle{vertex}=[draw,text=black]

  \node[vertex,double,circle] (S) {$0$};
  \node[vertex,double,circle] (E) [right of=S] {$1$};
  \node (EE) [right of=E] {};

  \draw [<-] (S) to node[auto] {} ++ (-1,0);
  \path %(S) edge   node  {$\varepsilon$}(A)
        (E) edge  [bend right=40] node [swap] {$b$}(S)
        (S) edge  [bend right=40] node [swap] {$a$}(E);
  \draw (E) to [out=375,in=345,looseness=8] node [above] {$b$} (E);
        %(Operations) to[out=-80, in=-90,looseness=2] (here)    to[out=90,in=80,looseness=2] (Operations);
        %(A) edge   node  {$\varepsilon$}(E);
\end{tikzpicture}

Строка $(ab)^\omega$ будет принята, строка $a(b^\omega)$ --- нет.
\end{exm}
\end{frame}

\begin{frame}{Строим автомат Бюхи для формулы: состояния}
\begin{itemize}
\item Раскроем сокращения записи (выразим $\Box$, $\vee$, $\rightarrow$ и $\Diamond$ через другие связки).
\item Рассмотрим $\mathcal{B}(\varphi)$ --- семейство всех подформул $\varphi$ (и их отрицаний,
с учётом $\varphi = \neg\neg\varphi$), 
образующих непротиворечивое максимальное множество. Скажем, для $a\vee\neg b$ это будет
$\{\{\neg a,\neg b,a\vee\neg b\},\{a,\neg b,a \vee\neg b\},
\{\neg a,b,\neg(a\vee\neg b)\},\{a,b,a\vee\neg b\}\}$
\item Поскольку множество содержит модальные операторы, непротиворечивость также должна
соответствовать условиям реализуемости: при рассмотрении подформулы $\varphi\mathcal{U}\psi$ должно быть
$$\psi\in B \Rightarrow \varphi\mathcal{U}\psi \in B\quad\quad
  \varphi\mathcal{U}\psi \in B \text{ и }\psi \notin B \Rightarrow \varphi \in B$$
%$$\varphi\mathcal{U}\psi = \psi\vee(\varphi\wedge\bigcirc(\varphi\mathcal{U}\psi))$$
\item Состояния автомата --- $B_n \in \mathcal{B}(\varphi)$.
\item Стрелки подписаны состоянием переменных $A$, и их может быть несколько одинаково
подписанных, поскольку в силу модального характера значение формулы не исчерпывается 
значением переменных.
\end{itemize}
\end{frame}

\begin{frame}{Строим автомат Бюхи для формулы: переходы}
Рассмотрим состояние $B$ и набор переменных $A \in \mathcal{P}(FV(\varphi))$.
\begin{itemize}
\item Рассмотрим $B': A = B\cup FV(\varphi)$ --- состояния, в которых
пропозициональные переменные соответствуют ожидаемому набору переменных.
\item Тогда $B' \in \delta(A,B)$, если и только если для каждого из $\psi\in B$
выполнено одно из следующих условий:
\begin{enumerate}
\item 
если $\psi$ --- не модальный оператор (здесь неявная рекурсия по структуре $\psi$), то $\psi\in B'$;
\item если $\psi\equiv\bigcirc\varphi$, то $\varphi\in B'$;
\item если $\psi \equiv \varphi_1\mathcal{U}\varphi_2$, то выполнен закон расширения для $\mathcal{U}$:
\begin{itemize}
\item либо $\varphi_2 \in B$ ($\mathcal{U}$ активирован в текущем состоянии).
\item либо $\varphi_1 \in B$ и $\varphi_1 \mathcal{U} \varphi_2 \in B'$
($\mathcal{U}$ будет активирован позже).
\end{itemize}
\end{enumerate}
\end{itemize}
\end{frame}

\begin{frame}{Строим автомат Бюхи для формулы: допускающие состояния}
Автомат для формулы $\varphi$:

$$\begin{array}{ll}Q &:= \mathcal{B}(\varphi) \\ Q_0 &:= \{ B\ |\ \varphi\in B, B\in Q\} \\
F_{\psi_1\mathcal{U}\psi_2} &:= \{B\ |\ \psi_1\mathcal{U}\psi_2 \not\in B \text{ или } \psi_2\in B\}, \\
\mathcal{F}&:=\{F_{\psi_1\mathcal{U}\psi_2}\}\end{array}$$

Идея в том, что автомат окажется в допускающем состоянии относительно $\psi_1\mathcal{U}\psi_2$:
\begin{itemize}
\item либо, когда формула $\psi_1\mathcal{U}\psi_2$ не нужна для результата;
\item либо в тот момент, когда соответствующая 
формула $\psi_1\mathcal{U}\psi_2$ <<активируется>> --- оператор меняет фокус восприятия
с ранее истинного $\psi_1$ на истинный $\psi_2$.
\end{itemize}
\end{frame}

\begin{frame}{Почему автомат Бюхи?}
Напомним, обобщённый автомат Бюхи принимает строку, если все допускающие множества состояния 
посещаются при обработке строки бесконечное количество раз. Соответственно:

\begin{itemize}
\item если автомат имеет $\mathcal{F} = \varnothing$, то он примет любую
бесконечную последовательность переходов.
\item если автомат построен для оператора $\mathcal{U}$, и оператор активируется
на шаге $k$ в последний раз --- значит, на шаге $k+1$ и на последующих шагах
данное выражение не будет истинным (для истинности $\varphi\mathcal{U}\psi$ требуется
наличия момента активации в будущем).
\end{itemize}
\end{frame}

\begin{frame}{Разрешимость задачи $TS \models \varphi$ в ЛТЛ}
%\begin{thm}Если система переходов конечна и не содержит терминальных состояний, то
%существует алгоритм, разрешающий данную задачу для любого $\varphi$. 
%\end{thm}

\begin{proof}[Идея алгоритма]
\begin{enumerate}
\item Построим обобщённый недетерминированный автомат Бюхи для формулы $\neg\varphi$
(принимает последовательность исполнения тогда и только тогда, когда она 
опровергает $\varphi$);
\item построим недетерминированный автомат по системе переходов $TS$;
\item построим их пересечение --- и преобразуем автомат в недетерменированный автомат Бюхи $T$
(с множеством допускающих состояний $F$).
\item проверим $\mathcal{L}_T = \varnothing$ (значит, $\varphi$ доказано), 
либо найдём контрпример --- последовательность исполнения, имеющая цикл, затрагивающий
состояние из $F$ (обход графа состояний; алгоритм заканчивается в силу конечности
множества состояний);
\item данная последовательность будет контрпримером к задаче $TS \models \varphi$.
\end{enumerate}
\end{proof}
\end{frame}

\begin{frame}{Пример реализации: SPIN}
\begin{itemize}
\item Один из первых инструментов (1991 год).
\item Используется специальный язык для описания алгоритмов/протоколов (Promela).
\item Язык позволяет формализовать параллельные вычисления.
\item Программа может быть вычислена в разных окружениях (например, случайное исполнение).
\item Также, к программе могут быть добавлены условия, которые либо будут доказаны --- либо
будет найден контрпример (контрпример будет предъявлен).
\end{itemize}
\end{frame}

\begin{frame}[fragile]{Пример программы на Promela: каковы возможные значения n?}
\begin{verbatim}
byte n = 0, finish = 0;
active [2] proctype P() {
    byte register, counter = 0;
    do :: counter = 10 -> break
       :: else ->
             register = n;
             register++;
             n = register;
             counter++
    od;
    finish++
}
active proctype WaitForFinish() {
    finish == 2;
    printf(“n = %d\n”, n)
}
\end{verbatim}
\end{frame}

%\begin{frame}{Сколько возможно ответов?}
%\begin{itemize}
%\pause\item На самом деле два.
%\end{itemize}
%\end{frame}
%\begin{frame}{}
%\end{frame}

\end{document}
